\documentclass[12pt]{article}

\title{E-CARDIAC Electronic CARDboard Illustrative Aid to Computation}
\author{Martín Osvaldo Santos Soto}
\date{\today}


%Packages
\usepackage[letterpaper,top=2cm,bottom=2cm,left=2cm,right=2cm]{geometry}
\usepackage[utf8]{inputenc}
\usepackage[spanish, activeacute]{babel}

%Subfiguras
\usepackage{caption}
\usepackage{subcaption}

% Para referencias 
\usepackage{hyperref}
\usepackage{apacite}
\usepackage{url}

\usepackage{graphicx} 

\begin{document}
	\maketitle
	
	Consideremos las primeras computadoras de la humanidad, ya hace miles de años los griegos tenian aparatos análogos que funcionaban con engranajes
	que podían hacer calculos complejos como predecir los eclipses de decenas de años en el futuro. Pero ¿podemos considerar a estos artefactos como "computadoras"?
	desde el origen etimologico de la palbra del latín "computare" que significa computar, calcular o evaluar podemos decir que si pues hacia calculos "automaticos"
	sin necesidad de intervención humana; pero quizá esa definición para los estandares modernos se quede corta, pues si bien es la base de lo que hace
	una computadora, calculos automaticos, algo más o igual de importante es la robustes de estos calculos, es decir que la "computadora" esté preparada
	para resolver más de un problema en particular(como lo hacia la primera de la que hablabamos) y que este lista para recibir una serie de instrucciones
	que pueda "entender" y de esa forma resolver un problema, es decir recibir un algoritmo* y ejecutarlo. Esto último sería la separación entre
	un motor de cálculo y una computadora para las definiciones actuales.
	
	Ya en tiempos modernos(1830) Charles Babbage con ayuda de Ada Lovelace crearon la "Máquina analitica" que es una idea teoríca sobre un artefacto capaz
	de resolver más que simples cálculos, utilizan ya el concepto de algoritmo para imaginar lo que 100 años después se convertirian en las computadoras que todos
	conocemos.
	
	Todas estás aportaciones iban encaminadas hacia un punto, la creación de un aparato capaz de resolver problemas, en principio simples cálculos, pero 
	en el camino la creatividad de los involucrados los llevo a buscar más que eso, que resolvieran problemas más complejos. Así es como llegamos
	a otro punto de inflexión en las computadoras, la famosa \textbf{Primera generación de computadoras}(1940-1956). En los años 40, difíciles años para los
	europeos, se dan grandes avances en la computación; tenemos la tesis de Church-Turing(1936) que establece los principios de la \textbf{computabilidad}, que
	sin ahondar mucho en los detalles es la definición de cuando alguna operación puede ser resuelta por una computadora, años después se da la creación
	de maquinas cada vez más complejas como la ENIAC en 1945 o la EDVAC años después ya con cálculos binarios y que recibia instrucciones por algoritmos
	a través de tarjetas perforadas.
	
	
	Es en este punto dónde llego CARDIAC de parte de los laboratorios Bell, ya existian maquinas complejas que usaban sistemas operativos y se empezaban
	a comercializar al publico en general en niveles muy pequeños, pero principalmente había más personas que trabjaban en computadoras, por ello la creación
	de CARDIAC como un manual y un artefacto teórico que sirviese para que las personas entendieran lo que sucedia detrás de una maquina que ya era tan compleja
	que no podían ver lo que los engranajes hacían y que de hecho ya tenían un lenguaje propio para comunicarse con las maquinas sin que sea el ensamblador.
	
	En este punto entra este trabajo, vamos a tomar esta maquina teórica que ayuda a entender los proceso de una computadora en una manera simple y vamos
	a modificarla de tal forma que pueda ser un modelo teórico que ayude a los usuarios de computadoras del siglo 21 a entender como funcionan estás maquinas
	que ahora tienen una ejecución de tareas concurrentes e incluso ejecución en paralelo. Podemos considerarla una actualización del trabajo original en el que
	además se añadirá un software especificamente diseñado en JAVA para emular el trabajo de está nueva computadora.
	

\end{document}