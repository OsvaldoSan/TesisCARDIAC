\documentclass[letterpaper,12pt,oneside]{article}
\usepackage[top=1in, left=1in, right=1in, bottom=1in]{geometry}
%-----------------------__--------
%https://es.overleaf.com/project/642e50d937469ff17340bdc4
% Tesis UNAM https://tex.stackexchange.com/questions/234265/unam-thesis-title-page-portada-tesis-unam

%\usepackage{parskip} % Add the parskip package


\usepackage{setspace} % Add the setspace package
\setstretch{1.5} % Set line spacing to double

\usepackage{pdfpages}
\usepackage{lipsum}

\usepackage[T1]{fontenc}
\usepackage[utf8]{inputenc}
\usepackage[spanish,es-nodecimaldot,es-tabla]{babel}

\usepackage{graphicx}
\usepackage{float}
\usepackage{tikz}
\usepackage{setspace}

%% para codigos
\usepackage{listings}


%Subfiguras
\usepackage{caption}
\usepackage{subcaption}

% Para referencias 
\usepackage{hyperref}
\usepackage{apacite}
\usepackage{url}

%\addbibresource{CARDIAC}


\title{Anteproyecto de titulación para licenciatura en Matemáticas Aplicadas y Computación}

\begin{document}

	%%\frontmatter
	%%\pagestyle{plain} % Set page style to "plain" for the front matter
    \maketitle
	
	\clearpage
	
	\section{Datos del estudiante}
	
	\begin{enumerate}
		\item Nombre : Martín Osvaldo Santos Soto
		\item Dirección : Sabino 71, Santa María La Ribera, Cuauhtémoc, Ciudad de México
		\item Número Telefónico : 7757573645 y 5577284216
		\item Número de Cuenta : 41809069-1
		\item Promedio : 8.64
		\item Generación : 2018-2021
		\item Periodo : 2024-2
		\item Carrera : Matemáticas Aplicadas y Computación
		\item Opción de titulación : Tesis
		\item E-Mail : osvaldosantos823@gmail.com
	\end{enumerate}
	
	\section{Datos del asesor}
		\begin{enumerate}
			\item Nombre :  Jorge Vasconcelos Santillán
		    \item Dirección :  Diamante 77 Col. Estrella. CDMX 07810
			\item Número Telefónico :  55 2748 2684
			\item Profesión : Ingeniero en Computación
		
		\end{enumerate}		

	
		\section{Titulo del trabajo}
		
		El titulo del trabajo será \textbf{CARDIAC: La evolución hacia un modelo concurrente y paralelo}.
		
		\section{Hipótesis}
		
		Se plantea como hipótesis que con el uso de un lenguaje de programación orientado a objetos, como lo es Java, se podrá
		crear una maquina virtual que pueda simular los cálculos que \textit{CARDIAC} realiza con cartón y papel. Así como
		incrementar la potencia de los cálculos que puede realizar, sin dejar de lado sus simplicidad; logrando
		desarrollar también en la misma maquina virtual versiones evolucionadas del modelo original que permitan
		concurrencia y paralelismo de procesos para poder ser una herramienta útil y sencilla para los estudiantes.
		
		
		\section{Objetivo del trabajo}
			
			El objetivo del proyecto es ser una guía didáctica para los estudiantes de los primeros semestres
			una computadora a nivel general, así como entender los sucesos históricos que confluyeron para la creación de está. Con ello
			ser capaces de explorar
			tanto los conceptos de computación concurrente y paralela como
			el concepto de sistema operativo y como no podríamos tener computación concurrente o paralela como
			la conocemos sin un sistema operativo.
		
		\section{Esquema o indice}
		
		La tesis estará dividida en los siguientes capítulos:
		
		\begin{enumerate}
			\item Introducción
			\item Capitulo 1 Historia de la computación
			\item Capitulo 2 Arquitectura básica de las computadoras
			\item Capitulo 3 Evolución del modelo
			\item Conclusiones
			\item Bibliografía
			\item Anexos
			
		\end{enumerate}
		
		
		
		\subsection{Capitulo 1 Historia de la computación}
			Esté capitulo como su nombre indica estará centrado en la historia de la computación, en como
			numerosos sucesos históricos confluyeron para la creación de lo que hoy conocemos como computadora, así
			como dar nombre a esas personas que dieron forma a la computación como la conocemos hoy en día.
			
			Empezando por los sucesos más antiguos y que quizá más de uno podrida pensar que no tienen relación con las
			primeras computadoras, llegando a ese limbo dónde una calculadora del momento era muy similar
			a una computadora, hasta el punto dónde definitivamente toman caminos distintos y la potencia de calculo
			de estás nuevas maquinas es llevada a otros niveles por la necesidades que se tenían. Exploramos las primeras
			computadoras que ya eran reconocidas como tales y vemos como al pasar de los años las mismas necesidades y
			creatividad de las personas hacen que las computadoras no paren de evolucionar.
			
			Vemos el nacimiento de los lenguajes de programación como consecuencia a la búsqueda de una programación
			más sencilla de estás maquinas que cada vez se volvían más complejas. Y como el desarrollo de los sistemas
			operativos hace que se de un salto en las capacidades de estás computadoras, pero sobre todo
			en la simplicidad que es para el usuario usarlas.
			
			Casi cerrando el capitulo encontramos la creación de modelos didácticos de computación para la enseñanza desarrollados
			en los años 60, cuando apenas empezaban a aparecer las computadoras en lugares públicos, en está sección se presenta por primera
			vez \textit{CARDIAC} y se explica la importancia que estos modelos didácticos tuvieron y tienen para el aprendizaje.
			
			Para el cierre tenemos un breve repaso sobre la actualidad de las computadoras para situarnos en la etapa
			desde la cual se está contando está historia.
			
		\subsection{Capitulo 2 Arquitectura básica de las computadoras}
		
			En esté capitulo se explicará como está constituida una computadora que tiene la arquitectura más común de hoy en día, la
			arquitectura von Neumann, repasando las características que debe tener una computadora para ser de está arquitectura
			y las ventajas que ofrece. También se revisará el sistema de arranque de una computadora y como se desarrollo para disminuir
			la interacción necesaria entre el usuario y la maquina; y precisamente en el tema de la interacción también
			se toca el tema del sistema operativo, explicando la necesidad que conllevo al desarrollo de esté proceso tan especial para una maquina.
			
			En el mismo capitulo se tratan los conceptos de la computación concurrente y paralela, explorando que son, como surgieron,
			las variedades que podemos encontrar y las complicaciones que puede generar tener una sistema de computación
			con estás características.
			
			Para cerrar se explica en su totalidad lo que es el modelo de computo didáctico  que se tomo de base para hacer está tesis,
			explorando su arquitectura, su lenguaje y el flujo de trabajo que se debe seguir con esté modelo.
		
		\subsection{Capitulo 3 Evolución del modelo}	
		
			Es el capitulo central de la tesis, los dos capitulos anteriores convergen en esté, pues aquí se exponen las tres versiones
			de \textit{CARDIAC}	desarrolladas. Empezando por el modelo original implementado en Java para sustituir los cálculos
			manuales por calculos que realiza la maquina viritual, pero permitiendo al usuario comprender como interactuan los
			elmentos dentro de la maquina para realizar tales calculos. 
			
			Continuando se encuentra la explicación de la primer evolución del modelo, que es la
			versión concurrente, y que explica no sólo el funcionamiento de la maquina virtual, sino los elementos
			de hardware y de software que se requirieron para
			lograr implementar la ejecución de procesos concurres en una computadora basada en CARDIAC, además de explicar
			el software desarrollado en el lenguaje de \textit{CARDIAC} para hacer posible tal ejecución.
			
			Para finalizar de la misma forma que se explica el modelo concurrente se explica el modelo paralelo y concurrente, aprovechando
			que a este punto muchos elementos del modelo han sido explicados se puede ser más ágil en la explicación y
			centrarse más en los casos de uso.
		
		
		\section{Bibliografía preliminar}
		
		

\end{document}

