\documentclass[12pt]{article}

\title{E-CARDIAC Electronic CARDboard Illustrative Aid to Computation}
\author{Martín Osvaldo Santos Soto}
\date{\today}


%Packages
\usepackage[letterpaper,top=2cm,bottom=2cm,left=2cm,right=2cm]{geometry}
\usepackage[utf8]{inputenc}
\usepackage[spanish, activeacute]{babel}

%Subfiguras
\usepackage{caption}
\usepackage{subcaption}

% Para referencias 
\usepackage{hyperref}
\usepackage{apacite}
\usepackage{url}

\usepackage{graphicx} 

\begin{document}
	\maketitle
	
	\section{ Antecedentess }
	Consideremos las primeras computadoras de la humanidad, ya hace miles de años los griegos tenían aparatos análogos que funcionaban con engranajes
	que podían hacer calculos complejos como predecir los eclipses de decenas de años en el futuro. Pero ¿podemos considerar a estos artefactos como "computadoras"?
	desde el origen etimologico de la palbra del latín "computare" que significa computar, calcular o evaluar podemos decir que si pues hacia calculos "automaticos"
	sin necesidad de intervención humana; pero quizá esa definición para los estandares modernos sea insuficiente, pues si bien es la base de lo que hace
	una computadora, calculos automaticos, se queda en un problema \textit{particular} y es justo lo que diferencia a las computadoras modernas de aquel origen
	que tuvieron como motores de cálculos, pues una computadora está preparada para resolver más de un problema en particular.
	
	Ya en tiempos modernos(1830) Charles Babbage con ayuda de Ada Lovelace crearon la "Máquina analitica" que es una idea teoríca sobre un artefacto capaz
	de resolver más que cálculos simples y particulares, utilizan ya el concepto de algoritmo para imaginar lo que 100 años después se convertirian en las 
	computadoras que todos conocemos.
	
	Si continuamos avanzando en el tiempo ya en la decada de los 30's hubo grandes avances con las maquinas de cálculo, dos principalmente en los
	estados unidos de America, ambas dando pasos hacia adelante en la mejora de los mecanismos usados para hacer computo. el "Difertential Analyzer" de
	Vonnevar bush podía resolver ecuaciones diferenciales complejas y la Mark I de Aiken ya era una maquina que aprovechaba los avances en el electromagnetismo
	sucedidos a inicio de siglo, creando así una maquina electromecanica que podía resolver operaciones más complejas y ya la versión Mark IV es considerada
	la primera calculadora a gran escala completamente automatica que existio.
	
	Justo en este punto empezamos a tener ya no sólo ideas avanzadas de las computadoras, sino maquinas reales que podían ayudar a solucionar problemas. Esto condujo*   	a la la teoría mátematica a empezar a mirar a las maquinas como una posible solución
	para resolver  \ţextbf{todos} los problemas que se pudieran formular, puesto que si se puede resolver teoricamente con un algoritmo, una lista de pasos
	a seguir, tal lista se le podría pasar a una maquina que estuviese preparada para seguir esos pasos. Es aquí donde el matematico Alan Turing se decide a crear
	\textit{teoricamente} dicha maquina capaz de resolver cualquier problema, pero el mismo
	en 1936 en su trabajo 
	"On Computable Numbres, with an Application to the Entscheidungsproblem" demuestra que tal maquina no puede existir, esto explicado superficialmente, se debe
	a que hay más números que algoritmos. Aunque demuestra que dicha maquina no puede existir se queda con la idea de crear una maquina capaz de 	    
	realizar cualquier cálculo que sea posible,
	es decir una maquina universal, que en la posteridad sería llamada "Maquina Universal de Turing"; la teoría de está maquina no es que pueda resolver cualquier
	problema que exista, sino que de todos los problemas que pueden ser resuletos por maquinas especializadas, la maquina universal los
	puede resolver, es más está maquina puede simular cualquier otra maquina y no se limita a resolver problemas aritmeticos. Para dicho trabajo fue importante 
	definir lo que la computabilidad significa.
	
	Entramos ahora en un apartado muy teoríco, pero necesario, ¿que es la computabilidad? en principio podemos decir que define a todo aquello que puede ser 
	procesado por una computadora, pero ¿que es todo eso? Bueno, para ello se requiere una teoría completa, \textbf{la teoria de la computabilidad} que
	básicamente difine aquellos problemas que pueden ser clasificados como "problemas computacionales" y aquellos que no. Los problemas computacionales se
	definen como aquellos que reciben una entrada(de datos) y devuelven una salida siempre, un ejemplo puede ser la expresción decimal finita de algún número real.
	
	Así que a esté punto podemos tener claro lo que es una computadora, una maquina capaz de resolver cualquier problema computacional, al menos en la teoría por que
	habrá situaciones en las que el diseño de la computadora esté teoricamente preparado para resolver un problema pero nos podemos encontrar con problemas fisicos
	como la falta de memoria o de velocidad de procesamiento que lo impiden. En estos puntos es dónde entran los avances en el hardware que se sucedieron en los
	años de la segunda guerra mundial y posteriores.
	
	
	En los años 40, difíciles años para los
	europeos, se dan grandes avances en la computación; tenemos la tesis de Church-Turing(1936) que establece los principios de la \textbf{computabilidad}, que
	sin ahondar mucho en los detalles es la definición de cuando alguna operación puede ser resuelta por una computadora, años después se da la creación
	de maquinas cada vez más complejas como la ENIAC en 1945 o la EDVAC años después ya con cálculos binarios y que recibia instrucciones por algoritmos
	a través de tarjetas perforadas.
	
	
	Es en este punto dónde llego CARDIAC de parte de los laboratorios Bell, ya existian maquinas complejas que usaban sistemas operativos y se empezaban
	a comercializar al publico en general en niveles muy pequeños, pero principalmente había más personas que trabjaban en computadoras, por ello la creación
	de CARDIAC como un manual y un artefacto teórico que sirviese para que las personas entendieran lo que sucedia detrás de una maquina que ya era tan compleja
	que no podían ver lo que los engranajes hacían y que de hecho ya tenían un lenguaje propio para comunicarse con las maquinas sin que sea el ensamblador.
	
	En este punto entra este trabajo, vamos a tomar esta maquina teórica que ayuda a entender los proceso de una computadora en una manera simple y vamos
	a modificarla de tal forma que pueda ser un modelo teórico que ayude a los usuarios de computadoras del siglo 21 a entender como funcionan estás maquinas
	que ahora tienen una ejecución de tareas concurrentes e incluso ejecución en paralelo. Podemos considerarla una actualización del trabajo original en el que
	además se añadirá un software especificamente diseñado en JAVA para emular el trabajo de está nueva computadora.
	
	
	Estás pruebas las podemos terminar aquí.
	

\end{document}
